\documentclass[12pt]{article}
\usepackage{epsfig}
\usepackage{graphicx}
\usepackage{a4}
\usepackage{amsmath}
\usepackage{latexsym}
\usepackage{cite}
%\usepackage{draftwatermark}
\usepackage{lineno}
%\usepackage{chngcntr}
%\linenumbers


\usepackage{color}
\usepackage{colordvi}

\graphicspath{{Pics/}}
\DeclareGraphicsExtensions{.eps,.ps}

\textheight 22.0cm \textwidth 16.5cm
\oddsidemargin -0.1cm \evensidemargin -0.1cm

\usepackage{pslatex}
\usepackage[latin1]{inputenc}
\usepackage[T1]{fontenc}
\usepackage{amssymb}
\usepackage{url}

\newcommand{\msbar}{$\overline{\text{MS}}\, $}

%
% ----------------------------------------------------------------------------------------------
%
\begin{document}

\begin{titlepage}
\noindent
Draft 0.1  \hfill 05 June 2019\\
\\
DESY AA-BBB %\hfill  2019\\
\\

\vspace{1.3cm}

\begin{center}
  {\bf 

\large

Extraction of parton distribution functions using recent LHCb and ALICE heavy-flavour measurements
  }
  \vspace{1.5cm}

  {\large
    PROSA Collaboration
  }\\

  \vspace{1.2cm}

\end{center}
 %\input{authors.tex}  
  \vspace{2.4cm}
\begin{center}
\large
{\bf Abstract}
\vspace{-0.2cm}
\end{center}
%%
..............................
%%
\vfill
\end{titlepage}


%
% ----------------------------------------------------------------------------------------------
%
\newpage

\section{Introduction}
\label{sect:intro}

\section{Details of the QCD analysis}
\label{sec:th}

The following data sets are used in the present analysis:
\begin{itemize}
    \item combined Neutral Current (NC) and Charged Current (CC) inclusive DIS cross sections by H1 and ZEUS~\cite{}
    \item combined charm and beauty NC DIS cross sections by H1 and ZEUS~\cite{}
    \item measurements of charm hadroproduction by LHCb at 5 TeV~\cite{}, 7 TeV~\cite{} and 13 TeV~\cite{}, and by ALICE at 5 TeV~\cite{} and 7 TeV~\cite{}
    \item measurements of beauty hadroproduction by LHCb at 7 TeV~\cite{}.
\end{itemize}
Compared to the previous PROSA analysis~\cite{}, this analysis uses more measurements of charm hadroproduction\footnote{Since the main objective of the present analysis is to constrain the gluon PDF at low values of $x$, there was no intention to include other LHC measurements of beauty hadroproduction.}
and using the final (and more precise) combined HERA data on inclusive NC and CC DIS, and heavy-quark production in NC DIS.

For the LHCb and ALICE data, the normalised cross sections, $\frac{{\rm d}\sigma}{{\rm d}y} / \frac{{\rm d}\sigma}{{\rm d}y_0}$, are calculated from the absolute cross sections and are used in the 
QCD analysis, with $\frac{{\rm d}\sigma}{{\rm d}y_0}$ being the cross section in the center bin, $3 < y < 3.5$, of 
the measured rapidity range in each $p_T$ bin (in particular, ALICE cross sections at $|y| < 0.5$ are normalised to LHCb cross sections at $3 < y < 3.5$). 
The advantage of using the normalised cross section, as was motivated in~\cite{}, is a significant
reduction of the scale dependence of the theoretical prediction, retaining the sensitivity of the cross sections to 
the PDFs. 

The treatment of experimental uncertainties for the HERA data follow the original publications~\cite{}.
For the ALICE and LHCb data, those uncertainties which are reported as single numbers (independent of $p_T$ and $y$ bins) are treated as correlated, and uncorrelated uncertainties are obtained by subtracting the correlated ones from the total uncertainty. Furthermore, for the normalised cross sections the uncorrelated uncertainty 
on $\frac{{\rm d}\sigma}{{\rm d}y_0}$ is propagated as a correlated uncertainty to the respective complementary 
rapidity bins. 

The analysis is performed at NLO (highest available order for heavy-quark hadroproduction) in the fixed-flavour-number scheme with three active flavours.
Theoretical calculations for the HERA data sets follow Ref.~\cite{}.
Theoretical calculations for heavy-quark hadroproduction follow the previous PROSA analysis~\cite{}, with the only change is that we are using the \msbar mass scheme.
The predictions are computed using the MNR code for single-particle inclusive distributions in the pole mass scheme, with the transition from the pole to the \msbar masses as described in~\cite{}. These calculations are interfaced in xFitter. The factorisation and renormalisation scales are chosen to be $\mu_r = \mu_f = \sqrt{4m_Q(m_Q)^2+p_T^2}$. The heavy-quark masses are free in the fit. The calculations are supplemented with phenomenological non-pertutbative fragmentation functions to describe the transition of heavy quarks into hadrons. The fragmentation of charm quarks into D-mesons is described by the Kartvelishvili function with $\alpha_K = 4.4 m 1.7$ as measured at HERA~\cite{},
while for the fragmentation of beauty quarks $\alpha_K = 11 m 3$ is used as measured at LEP~\cite{}.

The PDFs are parametrised at the starting scale $\mu^2_{f0} = 1.9$~GeV$^2$ of QCD evolution as:
\begin{equation}\begin{aligned}
xg(x) &= A_{g} x^{B_{g}}\,(1-x)^{C_{g}}\, (1 + F_{g} {\textrm{ln} x}),\\
u_\mathrm{v}(x) &= A_{u_\mathrm{v}}x^{B_{u_\mathrm{v}}}\,(1-x)^{C_{qu_\mathrm{v}}}\,(1+D_{u_\mathrm{v}}x) ,\\
d_\mathrm{v}(x) &= A_{d_\mathrm{v}}x^{B_{d_\mathrm{v}}}\,(1-x)^{C_{qd_\mathrm{v}}},\\
x\overline{\mathrm{U}}(x)&= A_{\overline{\mathrm{U}}}x^{B_{\overline{\mathrm{U}}}}\, (1-x)^{C_{\overline{\mathrm{U}}}}\, (1+D_{\overline{\mathrm{U}}}x), \\
x\overline{\mathrm{D}}(x)&= A_{\overline{\mathrm{D}}}x^{B_{\overline{\mathrm{D}}}}\, (1-x)^{C_{\overline{\mathrm{D}}}}.
\end{aligned}
\label{eq:dv}
\end{equation}
Here $xg(x)$ is the gluon distribution, $xqu_{\mathrm{v}}(x)$ and $xqd_{\mathrm{v}}(x)$ are the up and down quark valence quark distributions, respectively, and $x\overline{\mathrm{U}}(x)$ and $x\overline{\mathrm{D}}(x)$ are 
the up- and down-type antiquark distributions, respectively, assuming relations $x\overline{\mathrm{U}}(x) = xaqu(x)$ and $x\overline{\mathrm{D}}(x) = xd(x) + xs(x)$, with $xu(x)$, $xd(x)$, and $xs(x)$ are the up, down, and strange antiquark distributions, respectively.
The sea quark distribution is defined as $x\Sigma(x)=xu(x)+xd(x)+xs(x)$.
The normalisation parameters $A_{u_{\mathrm{v}}}$, $A_{d_\mathrm{v}}$, and $A_{g}$ are determined by the QCD sum rules.
Additional constraints $B_{\overline{\mathrm{U}}} = B_{\overline{\mathrm{D}}}$ and
$A_{\overline{\mathrm{U}}} = A_{\overline{\mathrm{D}}}(1 - f_{s})$ are imposed to ensure the same normalisation
for the $xu$ and $xd$ distributions as $x \to 0$.
The strangeness fraction $f_{s} = xs/( xd + xs)$ is fixed to
$f_{s}=0.4$ as in the HERAPDF2.0 analysis~\cite{Abramowicz:2015mha}.

The parameters in Eq.~(\ref{eq:dv}) are selected by first fitting with all polynomial parameters set to zero,
and then including them independently one at a time in the fit until the $\chi^2$ of the fit is improved.
The procedure is stopped when no further improvement is observed.
The $\chi^2$ definition used for the HERA DIS data follows that of Eq.~(32) in Ref.~\cite{Abramowicz:2015mha}.
For the heavy-quark normalised cross sections, the experimental uncertainties are treated as additive, and the treatment of the experimental uncertainties for the HERA DIS data follows the prescription given in Ref.~\cite{Abramowicz:2015mha}.

According to the general approach of the HERAPDF2.0 analysis~\cite{Abramowicz:2015mha}, 
the fit, model, and parametrisation uncertainties are taken into account.
Fit uncertainties are determined using the criterion of $\Delta\chi^2 = 1$.
Model uncertainties are determined by varying the strangeness fraction $0.3 \leq f_{s} \leq 0.5$, the value of $Q^2_{\text{min}}$ imposed on the HERA data, $2.5 \leq Q^2_\textrm{min}\leq 5.0\textrm{GeV}^2$, and the scales for heavy quark production.
The latter was varied up and down by a factor of two both simultaneously and independently for $\mu_r$ and $\mu_f$, and it was found that the simultaneous variation of $\mu_r$ and $\mu_f$ results in a larger PDF uncertainty. Thus, the simultaneous $\mu_r$ and $\mu_f$ variation is used as a PDF uncertainty eigenvector.
The parametrisation uncertainty is estimated by extending the functional form in Eq.~(\ref{eq:dv}) with additional polynomial parameters, which are added or removed one at a time and do not improve the $\chi^2$.
Additionally, the parametrisation uncertainty is determined by varying $1.6 < \mu_\mathrm{f0}^2 < 2.2$~GeV$^2$.
The parametrisation uncertainty is constructed as an envelope, built from the maximal differences between the PDFs or QCD parameters resulting from the central fit and all parametrisation variations.
The total uncertainty is obtained by adding the fit, model, and parametrisation uncertainties in quadrature.


\section*{Acknowledgements}

%  We would like to thank ...
The work of O.~Z. has been supported by Bundesministerium f\"ur Bildung und Forschung (contract 05H18GUCC1).

%%%%%%%%%%%%%%%%%%%%%%%%%%%%%%%%%%%%%%%%%
%%%%%%%%%%%%%%%%%%%%%%%%%%%%%%%%%%%%%%%%%

\bibliographystyle{plain}
\bibliography{prosa2019}

\end{document}


